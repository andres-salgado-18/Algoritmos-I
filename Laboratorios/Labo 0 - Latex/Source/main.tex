\documentclass{article}
\usepackage[utf8]{inputenc}
\usepackage{caratula}
\input{Algo1Macros.tex}

\title{Mi taller LaTeX}
\author{Andr�s S}
\date{Abril 2022}

\begin{document}
%Caratula
\titulo{Ejercicios LaTeX}
\subtitulo{Gu�a introductoria}
\fecha{09/04/22}
\materia{Algoritmos y Estructuras de Datos I}
\integrante{Salgado, Andr�s}{1439/21}{andres.salgado.1812@gmail.com}



\maketitle

\vspace{5mm} %5mm vertical space
\section{Ejercicio}
        \paragraph{El factorial de un entero positivo n se define como: n! = $\prod_{i=1}^ni$ \\
         El factorial de 5 es: 5! = $\prod_{i=1}^5i = 1 \times 2 \times 3 \times 4 \times 5 = 120$ \hfill \break}

\vspace{5mm} %5mm vertical space

\section{Ejercicio}
\subsection{Especificaci�n}
    \begin{proc}{factorial}{\In n: \ent , \Out resultado: \ent}{}{}
    \pre{n \geq 0}
    \post{({n = 0  \rightarrow resultado = 1}) \wedge (n > 0 \rightarrow resultado = \prod_{k=1}^nk)}

\vspace{5mm} %5mm vertical space

    \end{proc}
\section{Ejercicio}

    \pred{todosPrimos}{s \TLista{\ent}}{(\forall i:\ent)(0 \leq i < |s| \implicaLuego esPrimo(s[i])) }
\vspace{5mm} %5mm vertical space
    \pred{alMenosUnPrimo}{s \TLista{\ent}}{(\exists i:\ent)(0 \leq |s| \yLuego esPrimo(s[i])) }
\vspace{5mm} %5mm vertical space

\section{Ejercicio}

\aux{sumaPrimos}{s \TLista{\ent}}{\ent}{$\sum_{i=0}^{|s|-1}$ if $esPrimo(s[i])$ then $s[i]$ else 0 fi;}





\end{document}